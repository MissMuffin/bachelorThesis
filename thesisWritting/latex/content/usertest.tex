As stated in \Cref{introduction} the prototype's \gls{UI} is evaluated by iterative \gls{UI} testing. For this a user is given a device running the prototype and a set of tasks related to the features of the prototype. While the user executes these tasks their reactions while using the prototype are observed and instances of problems with the \gls{UI} noted. The user is then asked to summarize their experience in regards how easily they were able to use the features of the protoype and what elements of the interface they were confused about.

Ideally, user tests of the \gls{UI} design are performed throughout the developement process of a product, but time permitted for only one set of user tests during the developement of this thesis.

After the implementation of a working prototype in accordance with the requirements set at the beginning of this thesis, the interview participants were invited to test the prototype. One participant had no experience nor interest in using knitting pattern charts and was therefore not included in the prototype test.
The participants were asked to execute tasks derived from the requirements inside the prototype app during which their reactions were observed, with the main goal of determining the usability of the two pattern chart formats. After the completion of the tasks the user's feedback concerning the prototype was discussed. Users were asked to to note wether the prototype met the expections they expressed during the interview and were it failed to do so. The users were also asked to judge the overall usability --- the ease of use of the features, as well as the unambiguousness of the user interface. All user tests were held separately without communication between the participants. The results of these user tests are summarized in the following paragraphs.

Both participants were able to create and name a new pattern without problems. When first confronted with the default pattern in the row editor both expressed initial confusion about the shortened row format. After a brief explanation on what the number and symbol combination signified in the row format and how it would be displayed expanded when viewed in the grid format both participants were quickly able to work with the row editor and accurately produce results they were aiming for. For this they at first relied on switching to the grid editor to see how their changes in the row editor would play out --- they were only able to grasp entirely how the row editor functioned after seeing the changed they did to the pattern in the expaned grid format. The participants did not expect the editor to support the standard editor funtionalities such as selecting, copying, cutting and pasting text. 
Both participants had problems when asked for the first time to edit the pattern in the grid editor: they expected the same typing behaviour from the symbol keyboard that they encountered in the row editor. It took a few tries and an explanation of the toggling mechanics to enable them to successfully use the grid editor. In both tests the participants found and used the buttons to switch editors and save the pattern without problems. The same holds true for the menu entries to rename and delete the pattern currently open in the editor.
At the time of testing viewing the pattern in row format still had some bugs: larger patterns were not scrollable and the current row would not be highlighted upon increase or descrease of the row counter. The testing of the row viewer will there be disregarded for this user test iteration. All functions of the row counter and the grid viewer were used and understood by both participants from the start. 

Following the instructed testing the participants were asked to express their feedback concerning the prototype. Both participants expressed the desire to see a tutorial or introduction to the formats upon first use of the app since it was not clear from the beginning that there were two formats available to present a pattern chart. After being confronted with the row format on opening a pattern in the editor, the participants expected the grid editor kyeboard to behave like the one found in the row editor. They had problems understanding how to use the toggle symbols in the grid editor keyboard and were confused why nothing happend when they touched the grid. For the wish to have a symbol pre-selected was voiced, to indicate that symbols have to be selected to be set on the grid and to help the user understand that touching a cell leads to the selected symbol being set to that cell. One user also mentioned that the trash can icon for the button designed to erase a cell was misleading, they expected the button to delete the whole pattern --- an eraser icon would be better suited. Another problem was the missing background behind the line numbers in the grid editor, when the grid was scrolled it was hard to read the numbers. To improve this a solid background should be added to the line number sections. 

The row viewer did not fulfill the participants' expectation at all --- both agreed that at the very least the pattern needs to be scrollable and the current row should be indicated. The wish to set the current row in the viewer by tapping the current row number in the counter section was also expressed.

The participants agreed that the protoype met the expectations set by the preliminary interviews, excepting the row viewer. They compared the process of creating, editing and viewing a pattern chart to their current methods, modifying a spreadsheet to take the form of a knitting pattern chart, and found the prototype to be much easier and efficient to use. They deemed the ability to edit and view a pattern in two formats very valuable, since same stitch repeptitions could be quickly entered in the row format without the hassle of entering every stitch individually in the grid editor. The grid editor offered the ability to easily view the whole pattern and to spot and correct mistakes in it, as well as input more varied stitches in a pattern. Both participants preferred the prototype to their current pattern editors and viewers. One participant expressed the wish for a prototype supporting the same functions with colors instead of stitch symbols for the creation and viewing of colored pattern charts.

The current version of the prototype does not implement the aforementioned improvements and changes yet.