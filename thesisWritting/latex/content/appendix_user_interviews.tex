\chapter{User Interviews}

\section{Question catalogue}
\begin{itemize}
    \que{Wie viele Stücke haben Sie im vergangenen Jahr gestrickt?}
    \que{Für wen stricken Sie? Enkel, für sich selber?}
    \que{Gibt es bestimmte Stücke, die Sie häufig Stricken?}
    \que{Benutzen Sie bestimmte Techniken häufiger als andere?}
    \que{Wie wählen Sie eine Strickmuster? Selber ausdenken, suchen (online))}
    \que{Wie arbeiten Sie damit?}
    \que{Wie gehen Sie vor wenn Sie mit einer Strickmusterschematik arbeiten?}
    \que{Könnten Sie sich vorstellen ein Strickmuster von einem mobilen Gerät abzulesen?}
    \que{In welchem Format würden Sie dies gerne sehen? (audial, visuell)}
    \que{Wie würden Sie dem Gerät zu erkennen geben, dass eine Reihe fertig gestrickt wurde? (Sprachbefehl, Knopf drücken)}
    \que{(Verschiedene Strickmustertemplates zeigen und nach der Lesbarkeit fragen)}
    \que{Haben Sie schon einmal selber Strickmusterschematiken erstellt?}
    \que{Könnten Sie sich vorstellen, dies auf einem Handy zu tun?}
    \que{Wie würden Sie sich dabei die Eingabe vorstellen?}
    \que{Bei welchen Aspekten des Stricken könnten Sie sich eine App als hilfreiche Unterstützung vorstellen}
\end{itemize}

\section{Interview with Thilo Ilg}
\setcounter{answer}{0}
\begin{itemize}
	\ans{Hat das letzte mal 2009 gestrickt.}
	\ans{Hat nur für Schule gestrickt, hatte 6 Jahre lang einen Strickkurs.}
	\ans{Socken.}
	\ans{Nadelspiel.}
	\ans{Von Lehrkraft ausgesucht.}
	\ans{Hat noch nicht mit Musterschematik gearbeitet.}
	\ans{-}
	\ans{Ja.}
	\ans{Beides ok, bevorzugt visuell.}
	\ans{Findet Spracheingabe sehr nützlich, Hände sind voll beim Stricken.}
	\ans{Beide Ansichten sind gut, würde sich aber gestört fühlen bei breiten Mustern vom ständigen Scrollen und würde Landscape-Modus besser finden, da mehr Platz zum Lesen der Reihe.}
	\ans{Nein.}
	\ans{Ja.}
	\ans{Würde gerne Bereiche markieren können für eine Masche. Hätte gerne Funktrion um mehrere Zellen zu markieren und dann mit einem Maschensymbol zu befüllen. Klicken zum auswählen einer Zelle, zB. zum Bearbeiten. wenn Zelle markiert, nach Eingabe eines Symbols soll dann gleich zur nächsten Zelle gesprungen werden.}
	\ans{Würde gerne Bilder von dem fertigen Gestrickten sehen und schriftliche Anweisungen bevor er sich mit der Schematik befasst.
	Will für komplexe Muster auf jeden Fall Schematik haben, für simplere Muster eher nicht notwendig. 
	Hätte gerne Symbole in verschiedenen Farben für Sichtbarkeit.
	Wünscht sich Knopf um auf aktuelle Reihe und default Zoomstufe zu springen.}
\end{itemize}

\section{Interview with Nadine Kost}
\setcounter{answer}{0}
\begin{itemize}
	\ans{25.}
	\ans{Freunde und für den Eigenbedarf.}
	\ans{Fingerlose Handschuhe, Socken.}
	\ans{Bevorzugt Rundstricken.}
	\ans{Internet, würde gerne selber Muster schreiben, arbeitet am häufigsten mit schriftlichen Musteranweisungen.}
	\ans{Keine besondere Arbeitsweise.}
	\ans{Ausdrucken und mit einem Stift die vollendeten Reihen durchstreichen.}
	\ans{Ja.}
	\ans{Visuell.}
	\ans{Würde gerne nach der Vollendung einer Reihe einen Knopf drücken können.Dies soll auch ausserhalb der App möglich sein, zum Beispiel wie in Spotify mit einem Eintrag in der Notification bar oder mit einem Lockscreen widget.}
	\ans{Bevorzugt: Zeilenansicht mit Knopf für den Wechsel zwischen Zeilen- und Zellenansicht. Hätte gerne am Ende der Reihe die Anzahl der Maschen angezeigt.}
	\ans{Nein.}
	\ans{Ja, kann sich das besonders gut vorstellen für Farbmuster.}
	\ans{Am Anfang sollte man die Grösse des Musters wählen können.
    In einem Raster dieser Grösse soll dann bei Tap auf eine Zelle eine Auswahlansicht eingeblendet werden, aus der man Symbole für verschiedene Maschen wählen kann.
    Nach kurzer Überlegung: es wäre benutzerfreundlicher ein Symbol als aktiv zu kennzeichnen, welches dann bei Klick auf eine Zelle in diese eingetragen wird.}
	\ans{Beim Reihenzählen in einer Strickmusterschematik. Projektmanagement für Strickprojekte. Als ein Übersetzer von metrischen Einheiten von Nadelgrössen, Gewichten und Längen in imperiale und umgekehrt. Hätte ebenfalls gerne schriftliche Anweisungen in einer App wo man mit Knopfdruck auf nächste Anweisung springen könnte, zB. in Verbindung mit Reihenzähler.}
\end{itemize}

\section{Interview with Angela Thomas}
\setcounter{answer}{0}
\begin{itemize}
    \ans{49, das Meiste waren 72 einmal im Jahr. Strickt schon seit vielen Jahren, allerdings keine Muster(zB. Zopf) sondern nur Rechts-Links.}
    \ans{Grösstenteils für Bekannte.}
    \ans{Stulpen, Dreieckstücher.}
    \ans{Nein.}
    \ans{Internet, Strickzeitung, Muster durch Bekannte gelernt.}
    \ans{Keine Erfahrung mit Musterstricken, hat bisher nur Häkelmuster (Form) benutzt.}
    \ans{Für Häkelmuster: mit Stecknadel Reihe markieren.}
    \ans{Ja.}
    \ans{Hätte gerne eine Sprachausgabe der momentanen Reihe und würde diese  dann durch Knopfdruck wieder wiederholen lassen.}
    \ans{Sprachbefehl: durchaus denkbar.}
    \ans{Beide Ansichten wurden als wichtig gefunden, ein Wechsel zB per Knopf ist sowohl bei der Mustererstellung als auch in der Strickansicht gewünscht. Zeilenansicht ist für Kurzschrift, Zellen zum genaueren Betrachten des Musters.}
    \ans{Nein.}
    \ans{Ja.}
    \ans{In der Zeilenansicht, wobei dann zwischen Zeilen - und Zellenansicht gewechselt werden kann. Möchte nicht darauf achten zu müssen Zellen zu zählen, daher wird Zeileneingabe bevorzugt.}
    \ans{Erklärung und anschauliches Beispiel für einzeilne Maschen beim Stricken denkbar, Strick-/Häkelmuster auf dem Gerät mitnehmen (hat selber keine Smartphone, könnte sich das aber vorstellen). Bevorzugt schriftliche Anweisungen bei Mustern und braucht Text um eine Musterschematik zu verstehen.}
\end{itemize}