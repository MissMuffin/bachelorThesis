Nowadays we find ourselves in a very different world than even 20 years ago.
Following the rise of the mobile phone we got the smartphone, the handy pocket
computer for young and old. With technological advance came better processing
power, more storage for all the important pictures and videos in life, and the
big display to view everything in nearly lifelike colours. Together with it came
 the advent of the mobile application, more commonly referred to as an app. The
 many apps available are helpful tools to manage our daily lives and needs.
 Google’s distribution service for Android apps, the Google Play Store,
 surpassed the 1 million app mark in 2014.

But despite the many apps available, one is hard pressed to find apps concerned
with knitting, and even fewer ones that present a clear user interface or adhere
to modern design guidelines for apps. This thesis analyzed knitting related
Android apps on the Google Play Store. Most of the apps I found there do not
support the inputting or displaying of a knitting pattern chart.

The difficulty with displaying such a chart on a mobile device derives from the
size of the device and its screen, which is a far smaller medium than a sheet of
paper, on which pattern charts are normally printed. The charts are therefore too
big to be viewed easily inside an app. The goal for this thesis is to research
how a knitting pattern chart can be input and displayed on mobile Android
devices, including both phones and tablets.


