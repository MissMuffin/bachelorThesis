To create a first release version the more user feedback would needed to be implemented. The current version of the prototype only shows a console message when an error occurs during the export or saving of a pattern and and user feedback for successful deletion and the changing of a pattern name is missing. Currently the prototype's \gls{UI} is optimized for use on a Nexus 9 Android tablet, smaller screen sizes would need to be supported to guarantee a consistent \gls{UI} across all devices. 

The requirements (see \Cref{requirements}) that have not been fulfilled yet can be added to future versions of the app. Additionally, support of stitches that are wider than one cell can be added for knitting techniques that span across multiple stitches, e.g. the instruction to \gls{k2tog}. During the user tests one participant also wished for a repeat counter to be included in the viewer, for cases where a certain pattern has to be repeated, e.g. as is often done in scarves. Other features could be to control the app with voice commands or to have a pattern read aloud to the user.  

It is also possible to integrate the functions of the prototype into an app designed to manage everything connected to knitting projects. Such and app could allow the user to keep an inventory of all the needles and yarns in his possession, keep track of a shopping list for future projects and allow the input of written instructions. The option to add pictures to a pattern, either taken directly on the device or added from disk, as well as to share a pattern from inside the app, e.g. via E-mail or DropBox, would also fit well into such an app. 
