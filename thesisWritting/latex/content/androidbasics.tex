\begin{itemize}
\item basic structure android app
\item activities, menu, lifecycle, xml
\item actionbar
\item fragments, lifecycle, communication to parent activity and other fragments, xml
\item listview gridview adapter
\item view, canvas behind view, layout == viewgroup
\item resources (layout, drawable, menu, values)
\item storage, internal and external, read write permissions in manifest
\item gradle
\end{itemize}

\section{Basic structure of an Android App}
Android is an open source operating system for mobile phones and tables based on the Linus kernel (\cite{androiddef}). Android apps are distributed on Google Play, a service owned by Google. To build and Android app Google offers the Android \gls{SDK}, containing sample projects, necessary Android libraries and an Android emulator. Additionally, Google recommends to use the official Android \gls{IDE} Android Studio, which is based on IntelliJ IDEA and offers many useful tools, including testing frameworks, a \gls{GUI} for screen layouts, and the build tool Gradle (\cite{androidstudio}). The concepts and Android components discussed throughout this chapter are taken from the Android Developer reference (\cite{androidreference}), training (\cite{androidtraining}), and \gls{API} guides(\cite{androidguides}) found online.

An Android app consists of  at least one activity which can contain one or more fragments --- both activities and fragments are basic Android application components. An activity, in the Android context, is a Java class that extends the Activity class. It manages a screen and is needed to display any user interface. Fragments implement a specific user interface or behaviour and should, ideally, be modular so they can be reused with multiple activities and in different screen configurations. Both components have their own lifecycles and events during the apps runtime. Activities feature methods such as onCreate(), onStart(), onStop(), and onFinish(), which can be overwritten to implement logic that is executed at different points in it’s life. Same applies for a fragment, but where an activity can stand alone, a fragment always needs to be attached to an activity; it is connected to that activity's lifecycle --- is the activity stopped, the fragment will stop as well.
When containing fragments, the activity’s job is that of managing those fragments through getters and setters and orchestrating the communication between fragments, which is done with callbacks defined in the fragments.

Activities and Fragments can house ViewGroups and Views, which define different UI components for Android. Such a view would be a TextView, which displays one multiple lines of text. Viewgroups and views can be extended in custom classes to implement non-standard behaviour.

\section{Basic Components of an Android App}

\subsection{Activity}
The Activity class is needed to display any user interface and as such usually has a single purpose --- handling a login would be such a purpose. An app consists of one or more activities that are in some way connected to each other (\cite{activities_in_app}). An activity can embed multiple fragments --- those fragments then live in a ViewGroup inside the activity's own view hierarchy \cite{androidfragment}.  

\subsection{Actionbar}
The actionbar located at the top of an app and has several important functions. It displays the application name or the title of an activity, houses the action buttons, and the action overflow. Action buttons should contain the most commonly and important actions used in an app (\cite{actionbar}). The action overflow contains action buttons that are hidden from plain view, either because the actionbar was not wide enough to show all buttons or because of a deliberate design decision. Such a decision is usually made when the button in question is connected to an action that is rarely used, e.g. renaming something, or when the action has far-reaching consequences and shouldn't be near buttons that are used frequently, lest the user accidently hits it. Such an action could be the deletion of the currently edited pattern, in case of an knitting app. 

\subsection{Fragment}
The Fragment class ususally implements a specific user interface or behaviour and should, ideally, be modular, so that they can be reused within multiple activities or in different screen configurations. Just like an activity a fragment has its own lifecycle, see \reffigure{fig:fragment_lifecycle}.

\begin{wrapfigure}{R}{0.5\textwidth}
    \includegraphics[width=1.0\linewidth]{images/fragment_lifecycle.png}
   	\caption[The lifecycle of a fragment. \protect\newline{\small at: \url{https://developer.android.com/images/fragment_lifecycle.png} (last accessed: 2016-08-09)}]{The Fragment life cycle}
   	\label{fig:fragment_lifecycle}
\end{wrapfigure}

A fragment's creation is always embedded in an activity (\cite{androidfragment}) --- forcing the fragment to pause or stop alongside its parent activity's lifecycle. Fragments can also house viewgroups and views and are intented to function as interchangeable modules, e.g. as \gls{UI} modules for an app that runs on devices of varying sizes and that wants to present the user with a dynamic \gls{UI} fit to suit the screen size (see \reffigure{fig:fragments_uimodules}).

\begin{wrapfigure}{O}{0.5\textwidth}
    \includegraphics[width=1.0\linewidth]{images/fragments.png}
   	\caption[Two fragments of one activity and their layout on two different screen sizes. \protect\newline{\small at: \url{https://developer.android.com/images/fundamentals/fragments.png} (last accessed: 2016-08-09)}]{Two fragments of one activity and their layout on two different screen sizes} 
   	\label{fig:fragments_uimodules} 
\end{wrapfigure}

Android offers different fragment subclassse with predefined behavior, such as the DialogFragment class, that opens a fragment as a floating dialog by default (see \reffigure{fig:dialog_fragment}). \todo{add figure dialog fragment} Communication between different fragments needs to be handled by the activity, which manages the fragments. An activity communicates with a fragment by keeping a reference to the fragment and calling its public methods. On the other hand the fragment should not possess a reference to its parent activity --- instead the parent activity should implement a callback interface defined inside the fragment (\cite{fragment_event_callback}). A good practive to enforce the implementation of a fragments callback interface is to check for its existence when the fragment is attached to the activity --- example code proposed by the Android Developer guide concerning how to check for this can be seen in \reflisting{lst:callback_interface} \cite{fragment_event_callback}.

\begin{lstlisting}[language=JAVA, caption=Example code for enforcing the implementation of a callback interface, label=lst:callback_interface]
public static class FragmentA extends ListFragment {
    OnArticleSelectedListener mListener;
    ...
    @Override
    public void onAttach(Activity activity) {
        super.onAttach(activity);
        try {
            mListener = (OnArticleSelectedListener) activity;
        } catch (ClassCastException e) {
            throw new ClassCastException(activity.toString() + " must implement OnArticleSelectedListener");
        }
    }
    ...
}
\end{lstlisting}

\subsection{View}


\subsection{Resources}


\subsection{Storage}
\label{storage}


\subsection{Gradle}
